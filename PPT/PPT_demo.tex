% ============================================================
% 文件名: demo.tex
% 文档类:ctexbeamer (原生支持中文)
% 描述: R3S 主题演示文档
% ============================================================
% 1. 加载 ctexbeamer 类
% aspectratio=169: 16:9 比例
% fontset=fandol: 使用 Fandol 字体集 (包含宋体、黑体等)
% scheme=plain: 关闭 ctex 的一些特殊段落格式,更适合 Beamer
% \documentclass{ctexbeamer}
\documentclass[aspectratio=169, fontset=fandol, scheme=plain]{ctexbeamer} % 16:9 宽屏




% 2. 【关键】强制全局中文使用宋体
% ctexbeamer 默认正文是黑体 (sans-serif),我们需要改回宋体 (serif)
% \setCJKmainfont{FandolSong}      % 设置主要中文字体为宋体
% \setCJKsansfont{FandolSong}      % 强制无衬线体也用宋体 (为了配合 Beamer 默认行为)
% \setCJKmonofont{FandolFang}      % 代码用仿宋

% 3. 加载基础宏包
\usepackage{graphicx} % 插入图片
\usepackage{mwe}        % 测试图片
\usepackage{amsmath, amssymb}
\usepackage{amsfonts}   % 数学字体
\usepackage{mathrsfs} % 花体字母
\usepackage{eucal}     % 欧拉花体字母
% 4. 加载自定义主题
\usetheme{R3S}

% % ⚠️ 重要:如果你需要在导言区额外加载数学包,请放在 \usetheme 之后!
% % 这样可以让你的设置覆盖主题内的默认设置。
% \usepackage{mathtools} % 例如:如果你想用更高级的公式环境




% ============ 文档基本信息 ============
\title{高考数学备考研讨会}           % 主标题
\subtitle{第二轮复习策略分析}        % 副标题
\author{主讲人:RUINA 老师}          % 作者
\date{2026年2月26日}                 % 日期



\begin{document}
\footlinecolor{}
\maketitle % 生成封面页
\footlinecolor{dred}
% --------------------------------------------------------
% 1. 封面页 (Title Page)
% 使用 [plain] 选项去除默认的页眉页脚,完全由主题控制
% --------------------------------------------------------
% \begin{frame}[plain]
%     \titlepage
% \end{frame}

% --------------------------------------------------------
% 2. 目录页 (Table of Contents)
% 布局与内容页一致,自动提取 section 生成
% --------------------------------------------------------
% \begin{frame}
%     \frametitle{目录}
%     \tableofcontents
% \end{frame}

% --------------------------------------------------------
% 3. 内容页示例 (Content Pages)
% --------------------------------------------------------

\section{函数与导数} % 定义第一章

\begin{frame}{核心考点分析} % 定义当前页标题
    
    % 正文内容
    本节我们将深入探讨函数单调性与导数的关系。
    
    \begin{itemize}
        \item \textbf{单调性判定}:利用 $f'(x)$ 的正负判断。
        \item \textbf{极值点}:导数为 0 且左右异号的点。
        \item \textbf{最值问题}:闭区间上的端点与极值点比较。
    \end{itemize}
    
    \vspace{0.5cm}
    
    % 使用块环境展示重点
    \begin{block}{重要结论}
        若 $f'(x) > 0$ 在区间 $D$ 上恒成立,则 $f(x)$ 在 $D$ 上单调递增。
    \end{block}
    \begin{colorblock}{yellow!20}{num}
        test text
    \end{colorblock}
\end{frame}

\begin{frame}{典型例题讲解}
    
    已知函数 $f(x) = x^3 - 3ax + 1$。
    
    \begin{enumerate}
        \item 讨论 $f(x)$ 的单调性;
        \item 若 $f(x)$ 在 $x=1$ 处取得极小值,求 $a$ 的值。
    \end{enumerate}
    
    \vspace{0.5cm}
    \alert{注意}:分类讨论思想在含参函数中的应用。
\end{frame}

\section{解析几何} % 定义第二章

\begin{frame}{圆锥曲线性质}
    
    本章重点回顾椭圆、双曲线及抛物线的定义与几何性质。
    
    \begin{columns}
        \column{0.5\textwidth}
        \begin{itemize}
            \item 第一定义
            \item 第二定义
            \item 焦半径公式
        \end{itemize}
        
        \column{0.5\textwidth}
        \centering
        插入示意图
        % \includegraphics[width=0.8\linewidth]{example-image-b}
        % \captionof{figure}{圆锥曲线示意}
    \end{columns}
\end{frame}

\begin{frame}{综合练习}
    
    已知椭圆 $\frac{x^2}{a^2} + \frac{y^2}{b^2} = 1$,其中 $a > b > 0$。
    
    \begin{itemize}
        \item 求椭圆的离心率 $e$;
        \item 证明:对于任意点 $P$ 在椭圆上,$PF_1 + PF_2 = 2a$。
    \end{itemize}
    
\end{frame}

\begin{frame}{公式测试}
    $f(x) = 2x + \frac{1}{2} x^2$
    测试\begin{equation*}
        \mathrm{i}\,\hslash\frac{\partial}{\partial t} \Psi(\mathbf{r},t) =
        -\frac{\hslash^2}{2\,m}\nabla^2\Psi(\mathbf{r},t)
        + V(\mathbf{r})\Psi(\mathbf{r},t)
    \end{equation*}
    \begin{equation*}
        f(x) = 2x + \frac{1}{2} x^2
    \end{equation*}
\end{frame}


% --------------------------------------------------------
% 4. 结束页 (Final Page)
% 使用自定义的 finalframe 环境
% --------------------------------------------------------
% \begin{finalframe}
%     % 内容在 sty 文件中已定义,此处留空即可
%     % 也可以在这里添加具体的联系方式或二维码
%     \vspace{-1cm}
%     \includegraphics[width=2cm]{example-image-c} \\
%     \tiny 扫码获取更多资料
% \end{finalframe}

\footlinecolor{}
\backmatter % 生成结束页

\end{document}