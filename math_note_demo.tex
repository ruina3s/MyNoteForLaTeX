\documentclass{ctexart} 
% ctexbook  适合长文档:chapter、section、subsection、subsubsection
% ctexart   适合短文档:section、subsection、subsubsection

% ---------- 导言区字体配置 ----------

% 1. ctex 宏包(自动加载 fandol)
\usepackage[UTF8,fontset=fandol]{ctex}
\setCJKmainfont{FandolSong}
\setCJKsansfont{FandolHei}
\setCJKmonofont{FandolFang}
% 2. 西文字体:Computer Modern Unicode
\usepackage{fontspec}
\setmainfont{cmun}[
    Extension = .otf,
    UprightFont = *rm,
    ItalicFont = *ti,
    SlantedFont = *sl,
    BoldFont = *bx,
    BoldItalicFont = *bi,
    BoldSlantedFont = *bl,
]
\setsansfont{cmun}[
    Extension = .otf,
    UprightFont = *ss,
    ItalicFont = *si,
    BoldFont = *sx,
    BoldItalicFont = *so,
]
\setmonofont{cmun}[
    Extension = .otf,
    UprightFont = *btl,
    ItalicFont = *bto,
    BoldFont = *tb,
    BoldItalicFont = *tx,
]
% 3. 数学字体(传统方案,非 unicode-math)
\usepackage{amsmath,amssymb}
\usepackage{mathrsfs}   % 用于 \CMcal
\usepackage{eucal}      % 用于 \EuScript
% --------------------------------

% ---------- 页面设置 ----------
\usepackage{geometry} % 页面设置
\geometry{left=2.8cm,right=2.8cm,top=2.5cm,bottom=2.5cm} % 页边距
% ----------------------------

% ---------- 链接 ----------
\usepackage{hyperref} % 超链接支持
\hypersetup{
    colorlinks=true, % 启用彩色链接
    linkcolor=red, % 目录链接颜色
}
% -------------------------

% ---------- 颜色 ----------
\usepackage{xcolor}
% code 
\definecolor{codebg}{RGB}{31,31,31} % 代码背景色
\definecolor{code}{RGB}{204,204,204} % 代码基本颜色
\definecolor{codecom}{RGB}{73,222,76} % 代码注释颜色
\definecolor{codekeys}{RGB}{50,201,255} % 代码关键词颜色
% math
\definecolor{math-bd-blue}{HTML}{2196f3} % 数学边框蓝色
\definecolor{math-bg-blue}{HTML}{e3f2fd} % 数学背景蓝色
\definecolor{math-bd-green}{HTML}{4caf50} % 数学边框绿色
\definecolor{math-bg-green}{HTML}{e8f5e9} % 数学背景绿色
\definecolor{math-bd-orrange}{HTML}{ff9800} % 数学边框橙色
\definecolor{math-bg-orrange}{HTML}{fff3e0} % 数学背景橙色
\definecolor{math-bd-red}{HTML}{f44336} % 数学边框红色
\definecolor{math-bg-red}{HTML}{ffebee} % 数学背景红色
\definecolor{math-bd-violet}{HTML}{9c27b0} % 数学边框紫色
\definecolor{math-bg-violet}{HTML}{f3e5f5} % 数学背景紫色
\definecolor{math-bd-aoi}{HTML}{00bcd4} % 数学边框青色
\definecolor{math-bg-aoi}{HTML}{e0f7fa} % 数学背景青色
% -------------------------

% ---------- 表格与列表 ----------
\usepackage{booktabs}       % 三线表
\usepackage{tabularx}       % 自适应列宽
\usepackage{enumitem}       % 列表定制
\setlist{nosep, left=0pt}   % 紧凑列表
\usepackage{multirow}      % 跨行合并
% ------------------------------

% ---------- 插入图片 ----------
\usepackage{graphicx}       % 插图支持
% -----------------------------

% ---------- 代码环境 ----------
\usepackage{listings}       % 代码高亮
\lstset{
    backgroundcolor=\color{codebg}, % 背景色
    basicstyle=\ttfamily\small\color{code}, % 基本代码样式
    keywordstyle=\color{codekeys}, % 关键词颜色
    commentstyle=\color{codecom}, % 注释颜色
    numbers=left, % 行号位置
    numberstyle=\tiny\color{gray}, % 行号样式
    breaklines=true, % 自动换行
    frame=single, % 单线框
    captionpos=b, % 标题位置
    showstringspaces=false, % 不显示字符串中的空格
    extendedchars=false, % 支持中文
    literate= % 处理中文显示
        {LaTeX}{{\LaTeX}}2
        {TeX}{{\TeX}}2
}
% -----------------------------

% ---------- 定理、定义环境 ----------
\usepackage{tcolorbox}

\tcbuselibrary{theorems,skins} % 引入定理和皮肤库
% 定理环境模板
% \definecolor{myblue}{RGB}{0,100,200}
% \definecolor{theorembg}{RGB}{245,245,255}
% \newtcbtheorem[number within=section]{theorem}{定理}{ 
%     % number within=section编号
%     % no counter无编号
%     enhanced, % 启用增强功能
%     colback=theorembg, % 背景色
%     colframe=myblue, % 边框色
%     fonttitle=\bfseries, % 标题字体加粗
%     drop fuzzy shadow % 阴影效果
% }{thm}

% \newtcbtheorem[number within=section]{definition}{定义}{
%     enhanced,                    % 启用高级功能
%     colback=gray!5,             % 背景色:极浅灰
%     colframe=gray!30,           % 边框色:浅灰
%     boxrule=0pt,                % 外边框宽度为0(只留左边框)
%     leftrule=3pt,               % 左边框宽度(绿色竖线效果)
%     sharp corners,              % 直角(或改为 rounded corners 圆角)
%     fonttitle=\bfseries\sffamily\color{black},  % 标题字体
%     coltitle=black,             % 标题颜色
%     attach title to upper,       % ★ 标题连接到内容上方(行内)
%     separator sign={.},         % 标题与内容分隔符
%     after title={\ },     % 标题后添加空格
%     description delimiters parenthesis,  % 描述文字用括号
%     drop fuzzy shadow             % 模糊阴影效果
% }{def}


% 定义 --- 绿
\newtcbtheorem[number within=section]{dy}{定义}{ 
    enhanced, 
    colback=math-bg-green,
    colframe=math-bd-green,
    boxrule=0pt,
    leftrule=3pt,
    sharp corners,
    fonttitle=\bfseries,
    coltitle=black,
    attach title to upper,
    separator sign={.},
    after title={\ },
    description delimiters parenthesis,
    drop fuzzy shadow
}{def}
% 定理、引理 --- 蓝
\newtcbtheorem[number within=section]{dl}{定理}{ 
    enhanced, 
    colback=math-bg-blue,
    colframe=math-bd-blue,
    boxrule=0pt,
    leftrule=3pt,
    sharp corners,
    fonttitle=\bfseries,
    coltitle=black,
    attach title to upper,
    separator sign={.},
    after title={\ },
    description delimiters parenthesis,
    drop fuzzy shadow
}{thm}
% 例题、例子 --- 紫
\newtcbtheorem[number within=section]{lz}{例}{ 
    enhanced, 
    colback=math-bg-violet,
    colframe=math-bd-violet,
    boxrule=0pt,
    leftrule=3pt,
    sharp corners,
    fonttitle=\bfseries,
    coltitle=black,
    attach title to upper,
    separator sign={.},
    after title={\ },
    description delimiters parenthesis,
    drop fuzzy shadow
}{tip}
% 练习、作业 --- 青
\newtcbtheorem[number within=section]{lx}{练习}{ 
    enhanced, 
    colback=math-bg-aoi,
    colframe=math-bd-aoi,
    boxrule=0pt,
    leftrule=3pt,
    sharp corners,
    fonttitle=\bfseries,
    coltitle=black,
    attach title to upper,
    separator sign={.},
    after title={\ },
    description delimiters parenthesis,
    drop fuzzy shadow
}{exer}
% --- 红
\newtcbtheorem[number within=section]{cw}{错误}{ 
    enhanced, 
    colback=math-bg-red,
    colframe=math-bd-red,
    boxrule=0pt,
    leftrule=3pt,
    sharp corners,
    fonttitle=\bfseries,
    coltitle=black,
    attach title to upper,
    separator sign={.},
    after title={\ },
    description delimiters parenthesis,
    drop fuzzy shadow
}{warn}
% 提示、注意 --- 橙黄
\newtcbtheorem[number within=section]{ts}{提示}{ 
    enhanced, 
    colback=math-bg-orrange,
    colframe=math-bd-orrange,
    boxrule=0pt,
    leftrule=3pt,
    sharp corners,
    fonttitle=\bfseries,
    coltitle=black,
    attach title to upper,
    separator sign={.},
    after title={\ },
    description delimiters parenthesis,
    drop fuzzy shadow
}{tip}
% ----------------------------------

% ---------- 练习与答案环境 ----------
\usepackage{comment}
% 答案显示控制
\newif\ifshowanswers
\showanswerstrue
% 显示答案:\showanswerstrue
% 隐藏答案:\showanswersfalse
\ifshowanswers
    \newenvironment{solution}
        {\par\smallskip{\color{red}\bfseries 【解析】解:}}
        {\par\smallskip{\color{red}\bfseries 【解析】解:}} % 解答标题
        {\par\smallskip} % 解答结束
\else
    \excludecomment{solution} 
\fi
% 自定义练习环境
\newcounter{exercise}[section]
\renewcommand{\theexercise}{\thesection.\arabic{exercise}}
\newenvironment{exercise}
    {\par\vspace{0.3cm}\refstepcounter{exercise}
     \noindent\textbf{练习 \theexercise}\quad}
    {\par\vspace{0.2cm}}
% ----------------------------------

% ---------- 分栏 ----------
\usepackage{multicol} % 支持多栏
% -------------------------

% ---------- 公式快捷命令 ----------

% --------------------------------





% ctexbook放在前面会导致章节编号问题,ctexart放在前面则没有问题
\title{LaTeX 数学笔记模板} 
\author{RUINA}
\date{2026年2月12日}
% ---------- 正文 ----------
\begin{document}
% \frontmatter % 前置部分 (ctexart不支持)
\maketitle
\newpage
\tableofcontents % 目录
\newpage
% \mainmatter % 正文部分 (ctexart不支持)
\section{普通文本}
当第连的里大郭平屯杨略丑求清送事,句杀燕入本火土不陈云必大而不后司,亲召是疾反秦,
皇派真你斯无一不救杨侯反的同,土希把正尝不往而忧你作使马也的种,了长在但,光九非间洪松冒自君朗逃尘得谋头位,
入书尚办不的心回君到,于上葬极仅认不老苟作十下无说赏禀,不恨若韩。
\newline
Lorem ipsum dolor sit amet consectetur adipisicing elit. 
Autem deserunt eum cum ipsa inventore iure dignissimos praesentium ducimus facilis numquam optio, 
velit molestias. Quod nobis dolor quos itaque, cupiditate cumque!
\subsection{test}
\subsubsection{test1}

\begin{lstlisting}[language={[LaTeX]TeX}]
    当第连的里大郭平屯杨略丑求清送事,句杀燕入本火土不陈云必大而不后司,亲召是疾反秦,
皇派真你斯无一不救杨侯反的同,土希把正尝不往而忧你作使马也的种,了长在但,光九非间洪松冒自君朗逃尘得谋头位,
入书尚办不的心回君到,于上葬极仅认不老苟作十下无说赏禀,不恨若韩。
\newline
Lorem ipsum dolor sit amet consectetur adipisicing elit. 
Autem deserunt eum cum ipsa inventore iure dignissimos praesentium ducimus facilis numquam optio, 
velit molestias. Quod nobis dolor quos itaque, cupiditate cumque!
\end{lstlisting}



\newpage
\section{表格}
三线表
\begin{table}[htbp] % htbp用于定位
    \caption{表格演示}
    \centering %剧中
    \begin{tabular}{lccc}
        \toprule
        表头&列1&列2&列3 \\ % \\换行 &换列
        \midrule
        行1&text11&text12&text13 \\
        行2&text21&text22&text23 \\ 
        \bottomrule
    \end{tabular}        
\end{table}

\begin{center}
    \begin{tabular}{|c|c|c|}
        \hline
        1&2&3 \\ \hline
        4&5&6 \\ \hline
        7&8&9 \\ \hline
    \end{tabular}
\end{center}

\begin{center}
    \begin{tabular}{|c|c|c|}
        \hline
        1&2&3 \\ \hline
        \multicolumn{2}{|c|}{4} &5 \\ \hline
        6& \multicolumn{2}{|c|}{7} \\ \hline
        \multicolumn{3}{|c|}{8} \\ \hline
    \end{tabular}
\end{center}

\begin{center}
    \begin{tabular}{|l|c|c|c|}
        \toprule
        % 横向合并示例:表头跨3列
        \multicolumn{1}{c}{表头} & \multicolumn{3}{c}{合并的3列标题} \\
        \cmidrule(lr){2-4}  % 在第2-4列画横线
        表头 & 列1 & 列2 & 列3 \\
        \midrule
        
        % 纵向合并示例:行1-行2 的第一列合并
        \multirow{2}{*}{合并行} & text11 & text12 & text13 \\
                                & text21 & text22 & text23 \\
        
        \midrule
        
        % 横向+纵向同时合并的复杂示例
        \multirow{2}{*}{复杂合并} & \multicolumn{2}{c}{横向合并两列} & text13 \\
        \cmidrule(lr){2-3}
                                  & text21 & text22 & text23 \\
        
        \bottomrule
    \end{tabular}
\end{center}


\begin{lstlisting}[language={[LaTeX]TeX}]
    \begin{table}[htbp] % htbp用于定位
        \caption{表格演示}
        \centering %剧中
        \begin{tabular}{lccc}
            \toprule
            表头&列1&列2&列3 \\ % \\换行 &换列
            \midrule
            行1&text11&text12&text13 \\
            行2&text21&text22&text23 \\ 
            \bottomrule
        \end{tabular}        
    \end{table}
    
    \begin{center}
        \begin{tabular}{|c|c|c|}
            \hline
            1&2&3 \\ \hline
            4&5&6 \\ \hline
            7&8&9 \\ \hline
        \end{tabular}
    \end{center}
    
    \begin{center}
        \begin{tabular}{|c|c|c|}
            \hline
            1&2&3 \\ \hline
            \multicolumn{2}{|c|}{4} &5 \\ \hline
            6& \multicolumn{2}{|c|}{7} \\ \hline
            \multicolumn{3}{|c|}{8} \\ \hline
        \end{tabular}
    \end{center}

    \begin{tabular}{|l|c|c|c|}
        \toprule
        % 横向合并示例:表头跨3列
        \multicolumn{1}{c}{表头} & \multicolumn{3}{c}{合并的3列标题} \\
        \cmidrule(lr){2-4}  % 在第2-4列画横线
        表头 & 列1 & 列2 & 列3 \\
        \midrule
        
        % 纵向合并示例:行1-行2 的第一列合并
        \multirow{2}{*}{合并行} & text11 & text12 & text13 \\
                                & text21 & text22 & text23 \\
        
        \midrule
        
        % 横向+纵向同时合并的复杂示例
        \multirow{2}{*}{复杂合并} & \multicolumn{2}{c}{横向合并两列} & text13 \\
        \cmidrule(lr){2-3}
                                  & text21 & text22 & text23 \\
        
        \bottomrule
    \end{tabular}


\end{lstlisting}


\newpage
\section{图片}
\begin{center}
    % 测试图片过大所以调整大小
    \includegraphics[scale=0.2]{测试用JPG.jpg}
    \includegraphics[scale=0.2]{测试用PNG.png}      
\end{center}
\begin{lstlisting}[language={[LaTeX]TeX}]
    \begin{center}
        % 测试图片过大所以调整大小
        \includegraphics[scale=0.2]{测试用JPG.jpg}
        \includegraphics[scale=0.2]{测试用PNG.png}        
    \end{center}
\end{lstlisting}



\newpage
\section{列表}
    \begin{itemize}
        \item 战争
        \item 疾病
        \item 命运
        \item 死亡
    \end{itemize}

    当第连的里大郭平屯杨略丑求清送事,句杀燕入本火土不陈云必大而不后司,亲召是疾反秦,
皇派真你斯无一不救杨侯反的同,土希把正尝不往而忧你作使马也的种,了长在但,光九非间洪松冒自君朗逃尘得谋头位,
入书尚办不的心回君到,于上葬极仅认不老苟作十下无说赏禀,不恨若韩。

    \begin{itemize}
        \item test1
        \begin{itemize}
            \item 123
            \item 345
        \end{itemize}
        \item test2
    \end{itemize}
    
    文字分割

    \begin{enumerate}
        \item kaiser
        \item number
        \item lucy
    \end{enumerate}

\begin{lstlisting}[language={[LaTeX]TeX}]
    \begin{itemize}
        \item 战争
        \item 疾病
        \item 命运
        \item 死亡
    \end{itemize}

    当第连的里大郭平屯杨略丑求清送事,句杀燕入本火土不陈云必大而不后司,亲召是疾反秦,
皇派真你斯无一不救杨侯反的同,土希把正尝不往而忧你作使马也的种,了长在但,光九非间洪松冒自君朗逃尘得谋头位,
入书尚办不的心回君到,于上葬极仅认不老苟作十下无说赏禀,不恨若韩。

    \begin{itemize}
        \item test1
        \begin{itemize}
            \item 123
            \item 345
        \end{itemize}
        \item test2
    \end{itemize}
    
    文字分割

    \begin{enumerate}
        \item kaiser
        \item number
        \item lucy
    \end{enumerate}
\end{lstlisting}

\newpage
\section{代码}
% 旧版
\begin{verbatim}
test = {"lily":18,"lucy":19,"luck":20}

for key in test.keys():
    print(key)
    print(test[key])

for value in test.values():
    print(value)

for k,v in test.items():
    print(k,v)
\end{verbatim}

% 引入listings包
\begin{lstlisting}[language=python]
    test = {"lily":18,"lucy":19,"luck":20}

    for key in test.keys():
        print(key)
        print(test[key])

    for value in test.values():
        print(value)

    for k,v in test.items():
        print(k,v)
\end{lstlisting}

\newpage
\section{定理、定义等*}
%     定理类型1
%     \begin{theorem}{定理测试} 
%         这是一个测试文本
%     \end{theorem}

% \begin{lstlisting}[language={[LaTeX]TeX}]
%     定理类型1
%     \begin{theorem}{定理测试} 
%         这是一个测试文本
%     \end{theorem}
% \end{lstlisting}
    \begin{dy}{测试}
    :这是一个测试定义
    \end{dy}
    \begin{dl}{}
    :这是一个测试定理
    \end{dl}
    \begin{lz}{}
    :这是一个测试例子
    \end{lz}
    \begin{ts}{}
    :这是一个测试提示
    \end{ts}
    \begin{cw}{}
    :这是一个测试错误
    \end{cw}
    \begin{lx}{}
    :这是一个测试练习
    \end{lx}
\begin{lstlisting}[language={[LaTeX]TeX}]
    % 在环境内容开始前加一个字符就正常显示,不然会吞掉一个字符
    \begin{dy}{测试}
    :这是一个测试定义
    \end{dy}
    \begin{dl}{}
    :这是一个测试定理
    \end{dl}
    \begin{lz}{}
    :这是一个测试例子
    \end{lz}
    \begin{ts}{}
    :这是一个测试提示
    \end{ts}
    \begin{cw}{}
    :这是一个测试错误
    \end{cw}
    \begin{lx}{}
    :这是一个测试练习
    \end{lx}
\end{lstlisting}
\newpage
\section{公式*}
%行内
$ a_n \neq 0 $
%段落
\[
f(x)=a_n x^n + a_{n-1}x^{n-1}+\cdots + a_0
\]
\begin{lstlisting}[language={[LaTeX]TeX}]
    %行内
    $ a_n \neq 0 $
    %段落
    \[
    f(x)=a_n x^n + a_{n-1}x^{n-1}+\cdots + a_0
    \]
\end{lstlisting}

\begin{tabular}{|c|c|c|c|}
    \hline
    公式 & 代码 & 公式 & 代码 \\
    \hline
    $a_{b}$ & \verb|a_{b}| & $a^{b}$ & \verb|a^{b}| \\
    \hline
    $f'(x)$ & \verb|f'(x)| & $a_{b}^{c}$ & \verb|a_{b}^{c}| \\
    \hline
    $5/8,\frac{5}{8}$ & \verb|5/8,\frac{5}{8}| & $\sqrt[3]{2}$ & \verb|\sqrt[3]{2}| \\
    \hline   
\end{tabular}






\newpage
\section{练习与解答*}
演示:
\begin{exercise}
    求极限 $\displaystyle \lim_{x\to 0} \frac{\sin 3x}{x}$.
\end{exercise}
\begin{solution}
    利用重要极限,$\displaystyle \lim_{x\to 0} \frac{\sin 3x}{x}=3\lim_{x\to 0}\frac{\sin 3x}{3x}=3\times 1=3$.
\end{solution}
\begin{lstlisting}[language={[LaTeX]TeX}]
    \begin{exercise}
        求极限 $\displaystyle \lim_{x\to 0} \frac{\sin 3x}{x}$.
    \end{exercise}
    \begin{solution}
        利用重要极限,$\displaystyle \lim_{x\to 0} \frac{\sin 3x}{x}=3\lim_{x\to 0}\frac{\sin 3x}{3x}=3\times 1=3$.
    \end{solution}
\end{lstlisting}





\newpage
\section{分栏}
当第连的里大郭平屯杨略丑求清送事,句杀燕入本火土不陈云必大而不后司,亲召是疾反秦,
皇派真你斯无一不救杨侯反的同,土希把正尝不往而忧你作使马也的种,了长在但,光九非间洪松冒自君朗逃尘得谋头位,
入书尚办不的心回君到,于上葬极仅认不老苟作十下无说赏禀,不恨若韩。
\setlength{\columnseprule}{0.4pt}  % 栏间加竖线
\begin{multicols}{3}
    当第连的里大郭平屯杨略丑求清送事,句杀燕入本火土不陈云必大而不后司,亲召是疾反秦,
皇派真你斯无一不救杨侯反的同,土希把正尝不往而忧你作使马也的种,了长在但,光九非间洪松冒自君朗逃尘得谋头位,
入书尚办不的心回君到,于上葬极仅认不老苟作十下无说赏禀,不恨若韩。
\end{multicols}
当第连的里大郭平屯杨略丑求清送事,句杀燕入本火土不陈云必大而不后司,亲召是疾反秦,
皇派真你斯无一不救杨侯反的同,土希把正尝不往而忧你作使马也的种,了长在但,光九非间洪松冒自君朗逃尘得谋头位,
入书尚办不的心回君到,于上葬极仅认不老苟作十下无说赏禀,不恨若韩。


\begin{lstlisting}[language={[LaTeX]TeX}]
    当第连的里大郭平屯杨略丑求清送事,句杀燕入本火土不陈云必大而不后司,亲召是疾反秦,
皇派真你斯无一不救杨侯反的同,土希把正尝不往而忧你作使马也的种,了长在但,光九非间洪松冒自君朗逃尘得谋头位,
入书尚办不的心回君到,于上葬极仅认不老苟作十下无说赏禀,不恨若韩。
\setlength{\columnseprule}{0.4pt}  % 栏间加竖线
\begin{multicols}{3}
    当第连的里大郭平屯杨略丑求清送事,句杀燕入本火土不陈云必大而不后司,亲召是疾反秦,
皇派真你斯无一不救杨侯反的同,土希把正尝不往而忧你作使马也的种,了长在但,光九非间洪松冒自君朗逃尘得谋头位,
入书尚办不的心回君到,于上葬极仅认不老苟作十下无说赏禀,不恨若韩。
\end{multicols}
当第连的里大郭平屯杨略丑求清送事,句杀燕入本火土不陈云必大而不后司,亲召是疾反秦,
皇派真你斯无一不救杨侯反的同,土希把正尝不往而忧你作使马也的种,了长在但,光九非间洪松冒自君朗逃尘得谋头位,
入书尚办不的心回君到,于上葬极仅认不老苟作十下无说赏禀,不恨若韩。
\end{lstlisting}



\newpage
\section{绘图}
暂未学会,待补充

\newpage
\appendix % 附录
\section{颜色}
\begin{table}[htbp]
    \centering
    \caption{颜色模式比较}
    \begin{tabular}{l|ccc}
        \toprule
        模式 & 语法 & 用途 & 取值范围 \\
        \midrule
        Named & \verb|\color{red}| & 简单快速 & 预定义名称 \\
        RGB & \verb|\color[RGB]{255,0,0}| & 屏幕显示 & 0-255 \\
        rgb & \verb|\color[rgb]{1,0,0}| & 屏幕显示 & 0-1 \\
        HTML & \verb|\color[HTML]{FF0000}| & Web/设计 & \#RRGGBB \\
        CMYK & \verb|\color[cmyk]{0,1,1,0}| & 印刷 & 0-1 \\
        HSB & \verb|\color[hsb]{0,1,1}| & 色彩调整 & 0-1 \\
        Gray & \verb|\color[gray]{0.5}| & 灰度 & 0-1 \\
        \bottomrule
    \end{tabular}
\end{table}
% \backmatter % 后置部分 (ctexart不支持)

\end{document}